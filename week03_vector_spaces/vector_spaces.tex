\documentclass[fullscreen=true, bookmarks=true, hyperref={pdfencoding=unicode}]{beamer}

\usepackage[utf8]{inputenc}                                % Кодировка
\usepackage[english,russian]{babel}                        % Переносы
\usepackage{xcolor}                                        % Работа с цветом
\usepackage{amsmath,amssymb,amsfonts}                      % Символы АМО
\usepackage{graphicx}                                      % Графика
\usepackage[labelsep=period]{caption}                      % Разделитель в подписях к рисункам и таблицам
\usepackage{hhline}                                        % Для верстки линий в таблицах
\usepackage[upright]{fourier}
\usepackage{verbatim}
\usepackage{tikz}                                          % Для простых рисунков в документе
\usetikzlibrary{matrix,arrows,decorations.pathmorphing,shapes.geometric,calc,snakes,backgrounds}
\usepackage{fancybox}                                      % Пакет для отрисовки рамок
\usepackage{verbatim}                                      % Для вставки кода в презентацию
% \usepackage{animate}                                       % Для вставки видео в презентацию
\usepackage{xmpmulti}                                      % Для вставки gif в презентацию
\usepackage{multirow}
% \usepackage{media9}                                        % To include youtube video

\usepackage{colortbl}
% \usepackage{tcolorbox}

\graphicspath{{../images/}}                                % Путь до рисунков
\setbeamertemplate{caption}[numbered]                      % Включение нумерации рисунков

\definecolor{links}{HTML}{2A1B81}                          % blue for url links
\hypersetup{colorlinks,linkcolor=,urlcolor=links}          % nothing for others

\usetheme{Malmoe}
\usecolortheme{seahorse}

% l' unite
\newcommand{\myunit}{1 cm}
\tikzset{
    node style sp/.style={draw,circle,minimum size=\myunit},
    node style ge/.style={circle,minimum size=\myunit},
    arrow style mul/.style={draw,sloped,midway,fill=white},
    arrow style plus/.style={midway,sloped,fill=white},
}

% \setbeameroption{show notes}
\setbeameroption{hide notes}

\title{Lecture 3. Vector spaces, coordinates, rank}
\author{Aleks Avdiushenko}
\institute{Neapolis University Paphos}
\date{May 24, 2023}
\titlegraphic{\includegraphics[keepaspectratio,width=0.25\textwidth]{nup_logo.png}}

\begin{document}
%\unitlength=2mm

% выводим заглавие
\begin{frame}
\transdissolve[duration=0.2]
\titlepage
\end{frame}


\begin{frame}{Concrete vs Abstract Vector Spaces}
  \begin{enumerate}
    \item \textbf{Concrete Vector Spaces:}
    \begin{itemize}
      \item Represented by specific examples
      \pause  
      \item Examples: $\mathbb{R}^n$, $\mathbb{C}^n$, functions, polynomials
      \pause
      \item Elements are $n$-tuples, functions, or polynomials with specific algebraic operations
    \end{itemize}
    \pause
    \vspace{0.5cm}
    \item \textbf{Abstract Vector Spaces:}
    \begin{itemize}
      \item Defined by a \textbf{set of axioms}
      \pause
      \item Examples: any set $V$ with operations $+$ and $\cdot$ satisfying vector space axioms
      \pause
      \item Elements can be any mathematical objects as long as they follow the axioms of a vector space
    \end{itemize}
  \end{enumerate}

  \begin{block}{The main benefit of abstraction}
    Any vector space with \textbf{finite basis} is essentially $\mathbb{R}^n$, so you could use it.
  \end{block}
\end{frame}


\begin{frame}{Abstract Vector Space}
  An \textbf{abstract vector space} is a set $V$ along with two operations, addition $(+)$ and scalar multiplication $(\cdot)$, that satisfy the following axioms:
  \begin{enumerate}
    \item \textbf{Closure under addition:} $\forall u, v \in V,\ u + v \in V$
    \item \textbf{Commutativity of addition:} $\forall u, v \in V,\ u + v = v + u$
    \item \textbf{Associativity of addition:} $\forall u, v, w \in V,\ (u + v) + w = u + (v + w)$
    \item \textbf{Existence of additive identity:} $\exists 0 \in V$ such that $\forall v \in V,\ v + 0 = v$
    \item \textbf{Existence of additive inverse:} $\forall v \in V, \ \exists -v \in V$ such that $v + (-v) = 0$    
  \end{enumerate}
\end{frame}


\begin{frame}
  \begin{enumerate}
    \item[6.] \textbf{Closure under scalar multiplication:} $\forall c \in \mathbb{F},\ v \in V,\ c\cdot v \in V$
    \item[7.] \textbf{Associativity of scalar multiplication:} $\forall c, d \in \mathbb{F},\ v \in V,\ (cd)\cdot v = c\cdot(d\cdot v)$
    \item[8.] \textbf{Distributivity of scalar multiplication over addition:} $\forall c \in \mathbb{F},\ u, v \in V,\ c\cdot(u + v) = c\cdot u + c\cdot v$
    \item[9.] \textbf{Distributivity of scalar addition over scalar multiplication:} $\forall c, d \in \mathbb{F},\ v \in V,\ (c + d)\cdot v = c\cdot v + d\cdot v$
    \item[10.] \textbf{Identity for scalar multiplication:} $\forall v \in V,\ 1\cdot v = v$
  \end{enumerate}
\end{frame}


\begin{frame}
  \frametitle{Examples of ordinary vector spaces}
  \begin{enumerate}
    \item \textbf{Real coordinate space:} $\mathbb{R}^n = \{(x_1, x_2, \ldots, x_n)^T \mid x_i \in \mathbb{R}\}$
    \item \textbf{Complex coordinate space:} $\mathbb{C}^n = \{(z_1, z_2, \ldots, z_n)^T \mid z_i \in \mathbb{C}\}$
    \pause
    \item \textbf{Polynomials:} $P_n(\mathbb{F}) = \{a_0 + a_1x + a_2x^2 + \ldots + a_nx^n \mid a_i \in \mathbb{F}\}$
    \item \textbf{Functions:} $C(\mathbb{R}, \mathbb{F}) = \{f: \mathbb{R} \to \mathbb{F} \mid f \text{ is continuous}\}$
    \pause
    \item \textbf{Matrices:} $M_{mn}(\mathbb{F}) = \{A \mid A \text{ is an } m \times n \text{ matrix with entries from } \mathbb{F}\}$   
  \end{enumerate}
\end{frame}


\begin{frame}
  \frametitle{And one slide for Tropical Space}
  A \textit{tropical space} is a set equipped with two binary operations, referred to as addition and multiplication. 
  
  Let's denote the tropical space as $(T, \oplus, \odot)$, where $T$ is the set, and $\oplus$ and $\odot$ are two binary operations satisfying the following axioms:
  
  \begin{enumerate}
      \item[T1] $(T, \oplus)$ is a commutative monoid with identity element $\infty$
      \item[T2] $\oplus$ is idempotent, meaning that for all $x \in T$, $x \oplus x = x$
      \item[T3] $\odot$ distributes over $\oplus$, meaning that for all $x, y, z \in T$, $x \odot (y \oplus z) = (x \odot y) \oplus (x \odot z)$
  \end{enumerate}
  
  \vspace{1cm}
  In tropical mathematics, addition corresponds to taking a minimum and multiplication corresponds to usual addition:
  
  $a \oplus b = \min(a,b)$ and $a \odot b = a + b$
  
\end{frame}
  

\begin{frame}{Definition of Linear Independence}
  A set of vectors $\{v_1, v_2, \ldots, v_n\}$ in a vector space $V$ is said to be \emph{linearly independent} if no vector in the set can be expressed as a linear combination of the others. In other words, the vectors are linearly independent if the only solution to the equation
  
  \[
  \lambda_1v_1 + \lambda_2v_2 + \cdots + \lambda_nv_n = 0
  \]
  
  is the trivial solution, where $\lambda_1 = \lambda_2 = \cdots = \lambda_n = 0$.

  \pause\vspace{0.5cm}
  \begin{block}{Remarks}
    \begin{enumerate}
      \item $\{v_1\} \text{ is linearly dependent } \Rightarrow v_1 = 0$
      \item $\{v_1, v_2\} \text{ is linearly dependent } \Rightarrow v_1 = \lambda v_2$
    \end{enumerate}
  \end{block}
\end{frame}


\begin{frame}{Tasks}
  Let's consider the vector space of \textbf{Functions:} $$C(\mathbb{R}, \mathbb{R}) = \{f: \mathbb{R} \to \mathbb{R} \mid f \text{ is continuous}\}$$

  \pause
  \begin{block}{Task 1}
    $\sin x,\ \cos x \in C(\mathbb{R}, \mathbb{R})$

    Are they linearly dependent?
  \end{block}

  \pause
  \begin{block}{Task 2}

    $1, \sin x,\ \sin^2 x, \hdots,\ \sin^n x \in C(\mathbb{R}, \mathbb{R})$

    Are they linearly dependent?
  \end{block}

\end{frame}


\begin{frame}{Definition of Basis}
  A \emph{basis} for a vector space $V$ is a set of vectors $\mathcal{B} = \{v_1, v_2, \ldots, v_n\}$ with the following properties:
  
  \begin{enumerate}
      \item \textbf{Linear independence}: The vectors in $\mathcal{B}$ are linearly independent, i.e., no vector in the set can be expressed as a linear combination of the others
      \item \textbf{Span}: The vectors in $\mathcal{B}$ span the vector space $V$, i.e., every vector in $V$ can be expressed as a unique linear combination of the vectors in $\mathcal{B}$
  \end{enumerate}
  
  In other words, a basis is a linearly independent set of vectors that spans the entire vector space $V$
\end{frame}


\begin{frame}{Equivalent Properties of Basis}
  A set of vectors $\mathcal{B} = \{v_1, v_2, \ldots, v_n\}$ in a 
  vector space $V$ is a \emph{basis} if and only if at least one 
  of the following equivalent properties holds:
  
  \begin{enumerate}
      \item \textbf{Definition of Basis}: $\mathcal{B}$ is linearly independent and spans $V$
      \item \textbf{Maximal Linearly Independent Set}: $\mathcal{B}$ is a linearly 
      independent set and no other linearly independent set in $V$ contains 
      more vectors than $\mathcal{B}$
      \item \textbf{Minimal Spanning Set}: $\mathcal{B}$ spans $V$ and 
      no other spanning set in $V$ contains fewer vectors than $\mathcal{B}$
  \end{enumerate}
  
  \vspace{0.5cm}\pause
  These properties are equivalent, meaning that if one of them holds for a set of vectors, then the other two properties also hold.
\end{frame}


\begin{frame}{The sacred meaning of the basis}
  Lev $V$ be a vector space, and $\{v_1, v_2, \ldots, v_n\}$ is its basis, then

  $$\forall v \in V\ \exists! \lambda_1, \ldots, \lambda_n : v = \lambda_1v_1 + \ldots + \lambda_nv_n$$

  \pause
  \begin{block}{Consequence}
    {\small
    \begin{itemize}
      \item Function $f: \mathbb{R}^n \to V \text{ with basis } \mathcal{B} = \{v_1, v_2, \ldots, v_n\}$:
      $$f((a_1, \ldots, a_n)) = a_1v_1 + \ldots + a_nv_n$$ is a \textbf{one-to-one and onto} mapping
      between their elements
      \item It is \textbf{bijective} linear transformation and it preserves 
      the structure of vector addition and scalar multiplication
      \pause
      \item So \textbf{isomorphism} exists between $\mathbb{R}^n$ and $V$, 
      we use the notation $\mathbb{R}^n \cong V$
      \item Isomorphic vector spaces have the same algebraic properties, 
      such as dimension and basis cardinality, 
      and can be thought of as essentially the same space, 
      just represented in different ways
    \end{itemize}
    }
  \end{block}
\end{frame}


\begin{frame}{Vector Subspace}
  A \textbf{vector subspace} (or \textbf{linear subspace}) $W$ of 
  a vector space $V$ is a subset of $V$ that is closed 
  under vector addition and scalar multiplication. 
  
  \pause
  In other words, $W$ is a vector subspace of $V$ if it satisfies the following properties:

  \begin{enumerate}
      \item The zero vector $\vec{0}$ of $V$ is in $W$
      \item If $\vec{u}$ and $\vec{v}$ are elements of $W$, then their sum $\vec{u} + \vec{v}$ is also in $W$
      \item If $\vec{u}$ is an element of $W$ and $c$ is a scalar, then the scalar product $c\vec{u}$ is also in $W$
  \end{enumerate}

  If $W$ satisfies these properties, we can say that $W$ is a subspace 
  of $V$ and write $W \subseteq V$.

  \pause
  \begin{example}
    $\{y \in \mathbb{R}^n \mid Ay = 0\} \subseteq \mathbb{R}^n$
  \end{example}
\end{frame}


\begin{frame}
  \begin{block}{Useful remark}
    If $W \subseteq V$, then $\dim W \le \dim V$

    \hspace{3.5cm} $= \Leftrightarrow W = V$
  \end{block}

  \pause
  \begin{block}{Task 3}
    Find basis of the space $W = \{y \in \mathbb{R}^n \mid Ay = 0\}$
  \end{block}
  \pause
  \vspace{-20pt}
  \begin{eqnarray*}
    \begin{matrix} \boldsymbol{x}_1 &  & \boldsymbol{x}_3 &  & \end{matrix} \quad\ \\[-7pt]
  \left[\begin{matrix}
    1 & 1 & 0 & 2 & \multicolumn{1}{|c}{0} \\[-2pt]
    \cline{1-2}
    0 & 0 & \multicolumn{1}{|c}{1} & 5 & \multicolumn{1}{|c}{0} \\
    \cline{3-4}
  \end{matrix}\right] \\[-7pt]
  \begin{matrix} & x_2 &  & x_4 &  \end{matrix} \quad
  \end{eqnarray*} 
  $\dim W = \text{ number of free variables}$
  \begin{columns}
    \begin{column}{0.4\paperwidth}
      \pause
      $$\begin{cases}
        v_1 = (*\ 1\ *\ 0) \\
        v_2 = (*\ 0\ *\ 1)
      \end{cases}$$          
    \end{column}
    \begin{column}{0.4\paperwidth}
      \pause
      $$\begin{cases}
        v_1 = (-1\ 1\ 0\ 0) \\
        v_2 = (-2\ 0\ -5\ 1)
      \end{cases}$$          
    \end{column}
  \end{columns}

  \pause
  {\small
  \begin{enumerate}
      \item So $\{v_1, v_2\}$ is linearly independent and 
      \item for any solution $v = (a\ b\ c\ d): bv_1 + dv_2 = v$, 
      because vector $v-bv_1 - dv_2$ is solution of our system
  \end{enumerate}
  We have built the \textbf{fundamental system of solutions}.}
\end{frame}


\begin{frame}{Linear Span}
  Let $V$ be a vector space over a field $\mathbb{F}$, and 
  let $S = \{\vec{v}_1, \vec{v}_2, \dots, \vec{v}_k\}$ be a subset of $V$. 
  The \textbf{linear span} (or \textbf{linear shell}) of $S$, 
  denoted by $\operatorname{span}(S)$, is the set of 
  all linear combinations of the vectors in $S$. 

  \[
      \operatorname{span}(S) = \left\{\lambda_1\vec{v}_1 + \lambda_2\vec{v}_2 
      + \cdots + \lambda_k\vec{v}_k \mid \lambda_i \in \mathbb{F},\  
      1 \leq i \leq k\right\}
  \]

  \pause
  Properties of the linear span:
  \begin{itemize}
      \item $\operatorname{span}(S)$ is a subspace of $V$
      \item If $S \subseteq T \subseteq V$, then $\operatorname{span}(S) \subseteq \operatorname{span}(T)$
      \item The smallest subspace containing $S$ is $\operatorname{span}(S)$
  \end{itemize}
\end{frame}


\begin{frame}{Example}
  \begin{block}{Task 4}
    Given five vectors in $\mathbb{R}^3$. 
    Find basis from these vectors and expand the rest of the vectors over it.
  \end{block}
  \pause
  \begin{enumerate}
    \item A simple solution ``on the forehead'': 
    we sort through all the subsets of vectors in search of 
    a linearly independent one
    \pause
    \item After that, we express the remaining vectors in terms of 
    the found ones, each time solving the system of linear equations 
    with the right side equal to one of the remaining vectors
  \end{enumerate}

\end{frame}


\begin{frame}{Optimal solution}
  $$\left[\begin{matrix}
    1 & 1 & 5 & 1 & -1 \\
    3 & 2 & 12 & 1 & 1 \\
    2 & 1 & 7 & 1 & 0 
  \end{matrix}\right] \to \dots \to
  \left[\begin{matrix}
    1 & 0 & 2 & 0 & 1 \\
    \cline{1-1}
    0 & \multicolumn{1}{|c}{1} & 3 & 0 & 0 \\
    \cline{2-3}
    0 & 0 & 0 & \multicolumn{1}{|c}{1} & -2 \\
    \cline{4-5}
    \end{matrix}\right]$$
    \begin{itemize}
      \item Numbers of principal variables are numbers of basis vectors
      \pause
      \item For our case they are $\color{blue!60}v_1, v_2, v_4$
    \end{itemize}
    \begin{align*}
      \left[\begin{array}{>{\columncolor{blue!20}}c >{\columncolor{blue!20}}c c >{\columncolor{blue!20}}c c}
        1 & 1 & 5 & 1 & -1 \\
        3 & 2 & 12 & 1 & 1 \\
        2 & 1 & 7 & 1 & 0 
      \end{array}\right] \to \dots \to 
      &\left[\begin{array}{c c >{\columncolor{gray!20}}c c >{\columncolor{gray!20}}c}
        1 & 0 & 2 & 0 & 1 \\
        \cline{1-1}
        0 & \multicolumn{1}{|c}{1} & 3 & 0 & 0 \\
        \cline{2-3}
        0 & 0 & 0 & \multicolumn{1}{|c}{1} & -2 \\
        \cline{4-5}
        \end{array}\right] \\
        &v_3 = 2v_1 + 3v_2 + 0v_4 \\
        &v_5 = 1v_1 + 0v_2 - 2v_4
    \end{align*}
\end{frame}


{ % all template changes are local to this group.
    \setbeamertemplate{navigation symbols}{}
    \begin{frame}<article:0>[plain]
        \begin{tikzpicture}[remember picture,overlay]
            \node[at=(current page.center)] {
                \includegraphics[keepaspectratio,
                                 width=\paperwidth,
                                 height=\paperheight]{change_coordinates_meme.jpg}
            };
        \end{tikzpicture}
     \end{frame}
}


\begin{frame}{Coordinates: 2D Example}
    \begin{columns}
      \begin{column}{0.4\paperwidth}
        \begin{tikzpicture}
          \coordinate (O) at (0,0);
          \coordinate (A) at (1.5, 0);
          \coordinate (C) at (  0, 1);
          \coordinate (B) at ($(A) + (C)$);
    
          % Draw the axes 1
          \draw[->] (-0.5, 0) -- (2, 0) node[right] {$\vec{e_1}$};
          \draw[->] (0, -0.5) -- (0, 2) node[above] {$\vec{e_2}$};
    
          % Draw the axes 2
          \draw[->, blue] (0, 0) -- (1.9, 0.5) node[right] {$\vec{f_1}$};
          \draw[->, blue] (0, 0) -- (-0.5, 1.9) node[above] {$\vec{f_2}$};
          
          \draw[->, thick] (O) -- (B) node[midway, above] {$\vec{v}$};
          \draw[dashed] (B) -- (C);
          \draw[dashed] (B) -- (A);
        \end{tikzpicture}  
      \end{column}
      \begin{column}{0.4\paperwidth}
        \begin{align*}
          \vec{v} = a\vec{e_1} + b\vec{e_2} \\
          \vec{v} = c\vec{f_1} + d\vec{f_2}
        \end{align*}
          \end{column}
    \end{columns}
    
    \begin{center}
      How to link basis vectors?
    \end{center}

    \pause
    \begin{columns}
      \begin{column}{0.4\paperwidth}
          \begin{align*}
            \vec{f_1} = c_{11}\vec{e_1} + c_{21}\vec{e_2} = 
            (\vec{e_1}, \vec{e_2}) 
            \begin{pmatrix} c_{11} \\ c_{21} \end{pmatrix} \\
            \vec{f_2} = c_{12}\vec{e_1} + c_{22}\vec{e_2} =
            (\vec{e_1}, \vec{e_2}) 
            \begin{pmatrix} c_{12} \\ c_{22} \end{pmatrix}
          \end{align*}
      \end{column}
      \begin{column}{0.4\paperwidth}

        $$\Rightarrow \quad (\vec{f_1}, \vec{f_2}) = (\vec{e_1}, \vec{e_2})
        \begin{pmatrix} 
          c_{11} & c_{12} \\ 
          c_{21} & c_{22}
        \end{pmatrix}$$
      \end{column}
    \end{columns}
\end{frame}


\begin{frame}{Change of coordinates in general case}
  Let $V$ be vector spaces with bases 
  $\{\vec{e_1}, \ldots, \vec{e_n}\}$ and 
  $\{\vec{f_1}, \ldots, \vec{f_n}\}$. 

  $$(\vec{f_1}, \ldots, \vec{f_n}) = (\vec{e_1}, \ldots, \vec{e_n})
  \begin{pmatrix} 
     &   & \\ 
     & C & \\ 
     &   & 
  \end{pmatrix}$$

  The matrix $C$ is called \textbf{the change of basis matrix} 
  from the basis $\{\vec{e_1}, \ldots, \vec{e_n}\}$ 
  to the basis $\{\vec{f_1}, \ldots, \vec{f_n}\}$. It is square and reversible.

  \pause\vspace{1em}
  Given a vector $\vec{v} \in V$, 
  its coordinates with respect to the 
  bases are related by:

  $$[\vec{v}]_{\vec{f}} = C^{-1}[\vec{v}]_{\vec{e}}$$

  where $[\vec{v}]_{\vec{f}}$ denotes the coordinates of $\vec{v}$ 
  with respect to the basis $\{\vec{f_1}, \ldots, \vec{f_n}\}$, 
  and $[\vec{v}]_{\vec{e}}$ denotes the coordinates of $\vec{v}$ with 
  respect to the basis $\{\vec{e_1}, \ldots, \vec{e_n}\}$.
\end{frame}


\begin{frame}{Matrix Rank}
  \begin{definition}
    The \emph{rank} of a matrix $A \in M_{mn}(\mathbb{R})$ is denoted as $\operatorname{rank}(A)$
  \end{definition}

  \begin{enumerate}
    \item It is the maximum number of linearly independent \textbf{columns}
    \pause
    \item It is the maximum number of linearly independent \textbf{rows}
    \pause
    \item \textbf{Factorial rank}: $\min \{k \mid A = B\cdot C\}$, where  
    $A \in M_{mn},\  B \in M_{mk},\  C \in M_{kn}$
    \pause
    \item \textbf{Tensor rank}: $\min \{k \mid A = x_1y_1 + \ldots + x_ky_k\}$, 
    where $A \in M_{mn},\  x_i \in M_{m1},\  y_i \in M_{1n}$
    \pause
    \item \textbf{Minor rank} is the largest \textbf{order} of a non-singular ($\det \neq 0$)
    square submatrix in $A$
    \item $\operatorname{rank}(A) = \text{ number of {\bf principal} variables }$
  \end{enumerate}
\end{frame}


\begin{frame}{Examples}
  \begin{align*}
    &A \in M_{mn} \\
    &\operatorname{rank}(A) = 0 \Leftrightarrow A = 0 \\
    &\operatorname{rank}(A) = 1 \Leftrightarrow A = xy, \text{ where } x \in M_{m1},\  y \in M_{1n} \\
  \end{align*}
  \vspace{-1cm}
  \begin{block}{Question}
    How to find factorial rank? Skeleton decomposition!
  \end{block}
  \pause
  \begin{align*}
    \left[\begin{array}{>{\columncolor{blue!20}}c >{\columncolor{blue!20}}c c >{\columncolor{blue!20}}c c}
      1 & 1 & 5 & 1 & -1 \\
      3 & 2 & 12 & 1 & 1 \\
      2 & 1 & 7 & 1 & 0 
    \end{array}\right] &\to \dots \to 
    \left[\begin{array}{c c >{\columncolor{gray!20}}c c >{\columncolor{gray!20}}c}
      1 & 0 & 2 & 0 & 1 \\
      \cline{1-1}
      0 & \multicolumn{1}{|c}{1} & 3 & 0 & 0 \\
      \cline{2-3}
      0 & 0 & 0 & \multicolumn{1}{|c}{1} & -2 \\
      \cline{4-5}
      \end{array}\right] \\
      & \Rightarrow A = 
      \left[\begin{array}{>{\columncolor{blue!20}}c|>{\columncolor{blue!20}}c|>{\columncolor{blue!20}}c}
        &     &    \\
        v_1 & v_2 & v_4 \\
            &     &    
      \end{array}\right] \cdot C 
  \end{align*}
\end{frame}


\begin{frame}{Matrix Rank Properties}
  \begin{itemize}
    \item $\forall C \mid \det C \neq 0: \operatorname{rank}(A) = \operatorname{rank}(CA)$
    \pause
    \item \textbf{Rank of Sum:} $|\operatorname{rank}(A) - \operatorname{rank}(B)| \le \operatorname{rank}(A+B) \le \operatorname{rank}(A) + \operatorname{rank}(B)$
  \end{itemize}

  \vspace{1cm}
  \begin{block}{Task 5}
    Provide the examples of matrices 
    $\operatorname{rank}(A) = 3,\ \operatorname{rank}(B) = 2$,
    where we'll see all the possibilities.
  \end{block}
\end{frame}


\begin{frame}
\begin{itemize}  
  \item \textbf{Rank of Product:} $\operatorname{rank}(A) + \operatorname{rank}(B) - k \le \operatorname{rank}(AB) \le \min\{\operatorname{rank}(A), \operatorname{rank}(B)\}$
  
  where $k$ is common dimension
  \pause    
  \item \textbf{Rank-Nullity Theorem:} For an $m \times n$ matrix $A$, 
  the rank of $A$ plus the nullity of $A$ is equal to the number of 
  columns in $A$: $$\operatorname{rank}(A) + \operatorname{nullity}(A) = n$$
  
  {\small The nullity of a matrix $A$ is the dimension of its null space 
  (also known as the kernel). The null space of a matrix $A$ is the set of 
  all vectors $x$ that satisfy the equation $Ax = 0$.}
  
  \pause    
  \item \textbf{Submatrix Property:} The rank of a matrix is always greater than or equal to the rank of any of its submatrices: $\operatorname{rank}(A) \geq \operatorname{rank}(B)$, where $B$ is a submatrix of $A$.
\end{itemize}  

\end{frame}


\begin{frame}{Task 6}
  $$\operatorname{rank}\begin{pmatrix}
    A & 0 \\
    0 & D
  \end{pmatrix} = 
  \operatorname{rank}A + \operatorname{rank}D$$

  \vspace{1cm}
  Find the $\operatorname{rank}\begin{pmatrix}
    A & B \\
    5A & 7B
  \end{pmatrix}$.
\end{frame}


\begin{frame}{Task 7}
  \begin{align*}
    &A \in M_n(\mathbb{R}) \\
    &(\widehat{A})_{ij} = (C_{ij}) \text{, where } C_{ij} \text{ is cofactor} \\
    & \det \widehat{A} = \det A ^ {n-1} \\
    & \operatorname{rank}(\widehat{A}) - ?
  \end{align*}
 
  \pause
  $$\operatorname{rank}(\widehat{A}) = 
  \begin{cases}
    n,\ \operatorname{rank}(A) = n \\
    1,\ \operatorname{rank}(A) = n-1 \\
    0,\ \operatorname{rank}(A) < n-1
  \end{cases}$$

  \pause
  \begin{block}{Hint for the hardest case}
    $A \widehat{A} = \widehat{A} A = \det A \cdot E \Rightarrow A \widehat{A} = 0
    \Rightarrow \text{ rows of } \widehat{A} \subseteq \{y \in \mathbb{R}^n \mid Ay = 0\}
    \Rightarrow \dim = n - \operatorname{rank}(A) = 1$
  \end{block}

\end{frame}

\end{document}