\documentclass[fullscreen=true, bookmarks=true, hyperref={pdfencoding=unicode}]{beamer}

\usepackage[utf8]{inputenc}                                % Кодировка
\usepackage[english,russian]{babel}                        % Переносы
\usepackage{xcolor}                                        % Работа с цветом
\usepackage{amsmath,amssymb,amsfonts}                      % Символы АМО
\usepackage{graphicx}                                      % Графика
\usepackage[labelsep=period]{caption}                      % Разделитель в подписях к рисункам и таблицам
\usepackage{hhline}                                        % Для верстки линий в таблицах
\usepackage[upright]{fourier}
\usepackage{verbatim}
\usepackage{tikz}                                          % Для простых рисунков в документе
\usetikzlibrary{matrix,arrows,decorations.pathmorphing,
shapes.geometric,calc,snakes,backgrounds,arrows.meta,3d}
\usepackage{fancybox}                                      % Пакет для отрисовки рамок
\usepackage{verbatim}                                      % Для вставки кода в презентацию
\usepackage{xmpmulti}                                      % Для вставки gif в презентацию
\usepackage{multirow}

\usepackage{colortbl}
\usepackage{pgfplots}
\pgfplotsset{compat=1.16}
\usepgfplotslibrary{colormaps}

\graphicspath{{../images/}}                                % Путь до рисунков
\setbeamertemplate{caption}[numbered]                      % Включение нумерации рисунков

\definecolor{links}{HTML}{2A1B81}                          % blue for url links
\hypersetup{colorlinks,linkcolor=,urlcolor=links}          % nothing for others

\usetheme{Malmoe}
\usecolortheme{seahorse}

% l' unite
\newcommand{\myunit}{0.5 cm}
\tikzset{
    node style sp/.style={draw,circle,minimum size=\myunit},
    node style ge/.style={circle,minimum size=\myunit},
    arrow style mul/.style={draw,sloped,midway,fill=white},
    arrow style plus/.style={midway,sloped,fill=white},
}
\newcommand{\indep}{\perp \!\!\! \perp}

% \setbeameroption{show notes}
\setbeameroption{hide notes}

\title{Lecture 8. Random variable}
\author{Alex Avdiushenko}
\institute{Neapolis University Paphos}
\date{June 28, 2023}
\titlegraphic{\includegraphics[keepaspectratio,width=0.25\textwidth]{nup_logo.png}}

% \setbeamercolor{background canvas}{bg=gray}
% \setbeamercolor{normal text}{fg=white}
% \setbeamercolor{structure}{fg=mycyan}

\begin{document}

\begin{frame}
\transdissolve[duration=0.2]
\titlepage
\end{frame}


\begin{frame}{Recap and problems}
  \begin{itemize}
    \pause\item Still you can think of random variable as a variable whose possible values 
    are some outcomes of a random phenomenon (some \textbf{black box}, answering your questions).
    \pause\item We need to have pair $(\Omega, P)$, where $\Omega$ is all possible answers and $P$ is probability measure
    \pause\item But you may not want to work with all the $\Omega$ 
    \pause\item May be $\Omega$ is too informative, of complicated, or huge
    \pause\item Instead we can ask \textbf{black box} for some specific characteristics as weight or temperature
  \end{itemize}
\end{frame}


\begin{frame}
  \frametitle{Definition of a Random Variable}
  
  A \textbf{random variable} is a function that assigns a real number to each outcome in the sample space of a random experiment. 
  
  Formally, if $\Omega$ is the sample space of a random experiment, 
  a random variable $\xi$ is a function:
  
  \[
    \xi : \Omega \rightarrow \mathbb{R}
  \]
  
  In essence, a random variable provides a way to map the outcomes of a random process to numerical quantities.
  
\end{frame}


\begin{frame}{Examples}

  \begin{enumerate}
    \item For rolling a die $\Omega = \{1, 2, 3, 4, 5, 6\}$ random variable 
    $\xi(k) = k \mod 2$ shows us even results
    \item For $k$ fair coins $\Omega = \{(a_1, \dots, a_k) \mid a_i \in \{0, 1\}\}$ we can count Heads:
    $\xi((a_1, \dots, a_k)) = a_1 + \dots + a_k$
  \end{enumerate}

  \vspace{1cm}
  So any random variable $\xi(\omega)$ is \textbf{new black box} with 
  real numbers as samples and pair $(\mathbb{R}, P_\xi)$, where
  $$ P_\xi(A) = P(\xi \in A) = P(\{\omega\mid \xi(\omega) \in A \}) = P(\xi \text{ shows value from } A)$$

  $\xi^{-1}(A) = \{\omega\mid \xi(\omega) \in A \}$
\end{frame}


\begin{frame}
  So we want to study different measures on $\mathbb{R}$.

  \pause
  \begin{block}{Question}
    What probability measures can be on $\mathbb{R}$ at all?
  \end{block}  

  \pause Recall, that $P: 2^\mathbb{R} \to [0, 1]$ and 
  for any event $A \mapsto P(A)$
  \begin{itemize}
    \pause\item $P(\emptyset) = 0,\ P(\mathbb{R}) = 1$
    \pause\item $A_1, A_2, \ldots, A_n, \ldots \subseteq \mathbb{R},\  
    A_i \cap A_j = \emptyset$
    $$ P(\bigcup\limits_{i=1}^{\infty} A_i) = \sum\limits_{i=1}^{\infty} P(A_i) $$
  \end{itemize}

  \pause
  {\small It can be proven mathematically, that sufficently to know 
  only $P\left((a, b]\right)$ for all $a, b \in \mathbb{R}$
  $\Rightarrow$ there is general procedure, 
  which builds measure $P(A), A \subseteq \mathbb{R}$ 
  (it is called construction of the outer Lebesgue measure, 
  and of course here goes $\sigma$-algebras and so on)}
\end{frame}


\begin{frame}{Cumulative Distribution Function}
  Actually we need to know only $P\left((-\infty, x]\right)$ as 
  $$ P\left((a, b]\right) = P\left((-\infty, b]\right) - P\left((-\infty, a]\right) $$

  \pause
  \begin{block}{Definition}
    Cumulative Distribution Function (CDF) of 
    $\xi$ is $$F_\xi(x) = P_\xi\left((-\infty, x]\right) = P(\xi \leq x)$$
  \end{block}  

  \pause
  Vice versa some $F: \mathbb{R} \to \mathbb{R}$ is distribution function iff
  \begin{enumerate}
    \pause\item $x \leq y \Rightarrow F(x) \leq F(y)$
    \pause\item $\lim\limits_{x\to -\infty} F(x) = 0,\ \lim\limits_{x\to \infty} F(x) = 1$
    \pause\item $\lim\limits_{t > 0, t \to 0} F(x+t) = F(x)$
  \end{enumerate}

\end{frame}


\begin{frame}
  \frametitle{Examples}

  \begin{enumerate}
    \item Consider rolling a die: $\Omega = \{1, 2, 3, 4, 5, 6\}$, 
    random variable $\xi(k) = k \mod 2$ shows us even results
    \pause\item $\Omega = [0, 1],\ A \subseteq [0, 1],\ P(A) = \int\limits_{A}1dx = \text{length of A}$
  \end{enumerate}

  \centering
  \pause
  \begin{columns}
    \begin{column}{0.3\textwidth}
      \begin{tikzpicture}
        \draw[->] (-1,0) -- (2,0) node[right] {$\xi$};
        \draw[->] (0,-0.5) -- (0,2) node[above] {$F^1_\xi(x)$};
        \foreach \x in {0,1} {
            \draw (\x+0.08, -0.2) -- (\x+0.08, -0.2) node [left] {\x};
        }
          
        % CDF
        \draw[red] (-1,0) -- (0,0);
        \draw[red] (0,0.5) -- (1,0.5);  
        \draw[red] (1,1) -- (2,1);
        \filldraw[red] (0,0.5) circle (1pt);
        \filldraw[red] (1,1) circle (1pt);
  
        \draw[dashed] (0,1) -- (1,1) -- (1,0);
      \end{tikzpicture}        
    \end{column}
    \begin{column}{0.3\textwidth}
      \begin{tikzpicture}
        \draw[->] (-1,0) -- (2,0) node[right] {$\xi$};
        \draw[->] (0,-0.5) -- (0,2) node[above] {$F^2_\xi(x)$};
        \foreach \x in {0,1} {
            \draw (\x+0.08, -0.2) -- (\x+0.08, -0.2) node [left] {\x};
        }
          
        % CDF
        \draw[red] (-1,0) -- (0,0);
        \draw[red] (0,0) -- (1,1);  
        \draw[red] (1,1) -- (2,1);
        % \filldraw[red] (0,0.5) circle (1pt);
        % \filldraw[red] (1,1) circle (1pt);
  
        \draw[dashed] (0,1) -- (1,1) -- (1,0);
      \end{tikzpicture}        
    \end{column}
  \end{columns}
\end{frame}


\begin{frame}{Link between plots of $\xi(\omega)$ and $F_\xi(x)$}
  \begin{columns}
    \begin{column}{0.5\textwidth}
      \begin{tikzpicture}
        \draw[<->] (-2,0) node[left] {$F_\xi(x)$} -- (2,0) node[right] {$\Omega$};
        \draw[->] (0,-2.5) -- (0,2.5) node[above] {$\xi(\omega)$};
        \draw[dashed] (-1,-2.5) -- (-1,2.5);
    
        \coordinate [label={0}] () at (-0.14, -0.45);
        \coordinate [label={1}] () at (-1.14, -0.45);
          
        % Function \xi(x)
        \draw[blue, thick, domain=-0.96:0.96, smooth] plot (\x+1, {50*((\x^3)/3 - (\x^2)/2 + 0.21*\x))});
        % Function F_\xi(x)
        \draw[red, thick, domain=0.12:0.88, smooth] plot (-\x, {ln(\x/(1-\x))});
    
        % \draw[dashed] (0,1) -- (1,1) -- (1,0);
      \end{tikzpicture}                  
    \end{column}
    \begin{column}{0.5\textwidth}
      \pause
      So the vertical $\xi(\omega)$-axis is x-axis for CDF $F_\xi(x)$

      \pause\vspace{0.5cm}
      Moreover, different random variables $\xi$ may correspond to one and the same CDF. 
      For instance, you could reverse the $\Omega$-axis.
    \end{column}    
  \end{columns}

  \pause
  \begin{block}{Question}
    How to find CDF break points on this plot?
  \end{block}
\end{frame}


\begin{frame}{Probability atoms}
  \begin{columns}
    \begin{column}{0.5\textwidth}
      \begin{tikzpicture}[scale=1.0]
        % axes
        \draw[->] (-1,0) -- (4,0) node[right] {$x$};
        \draw[->] (0,-0.5) -- (0,3) node[above] {$F_\xi(x)$};
        
        % ticks
        \foreach \x in {1,2,3} {
            \draw (\x, 0.1) -- (\x, -0.1) node [below] {\x};
        }
        \draw (-0.1, 2) -- (0.1, 2) node [right] {1};
        \coordinate [label={0}] () at (-0.14, -0.43);

        \draw (2.6, 0.1) -- (2.6, -0.1) node [below] {$c$};
        \draw[dashed] (2.6, 0) -- (2.6, 1.1);

        % Function F_\xi(x)
        \draw[red, thick, domain=-1:1, smooth] plot (\x, {exp(\x/15)-0.8});
    
        % \draw[thick, red] (-0.5,0) -- (1,0);
        
        \draw[thick, red] (1,0.6) -- (2,0.6) -- (3,1.4);
        \draw[thick, red] (3,2) -- (4,2);
        
        % points
        \draw[red, fill=white] (1,{exp(1/15)-0.8}) circle (1.5pt);
        \draw[red, fill=white] (3,1.4) circle (1.5pt);
        
        \draw[red, fill=red] (1,0.6) circle (1.5pt);
        \draw[red, fill=red] (3,2) circle (1.5pt);
      \end{tikzpicture}    
    \end{column}
    \begin{column}{0.5\textwidth}
      \pause
      $P(\{c\}) = ?,\ P(1) = ?$
      
      \pause
      \begin{align*}
        P(\{c\}) = \lim\limits_{\delta \to 0} P\left((c-\delta, c+\delta]\right) = \\
        = \lim\limits_{\delta \to 0} \left(F(c+\delta) - F(c-\delta)\right) = 0
      \end{align*}
      
      \pause
      $P(1) = F(1) - F(1-) = \text{size of the jump discontinuity}$
    \end{column}
  \end{columns}
\end{frame}


\begin{frame}
  \frametitle{Classification of measures}

  \centering
  \begin{tabular}{|c|c|}
    \hline
    \textbf{Mathematics} & \textbf{Engineers} \\
    \hline
    Discrete &  Discrete \\
    Continuous &  -- \\
    Absolutely Continuous &  Continuous \\
    \hline
  \end{tabular}

  \pause\vspace{1cm}
  We are using the engineers terminology.
  
  \begin{enumerate}
    \pause\item Discrete consists of atoms $a_i$ only: $\sum\limits_{i=1}^\infty p_i = 1$
    \pause\item Continuous is defined by its density function $p(x) > 0$:
    $$\int\limits_{\mathbb{R}} p(x)dx = 1,\quad F^\prime(x) = p(x),\quad F(x) = \int\limits_{-\infty}^x p(t)dt$$
  \end{enumerate}
\end{frame}


\begin{frame}
  \frametitle{Cantor Function}
  
  The \textit{Cantor function}, also known as the \textit{Devil's Staircase}, 
  is a peculiar function that was discovered by Georg Cantor. 
  
  \begin{itemize}
      \pause\item It is defined on the interval $[0, 1]$
      \pause\item It is constant on the set of numbers that are not in the Cantor set, 
      which has Lebesgue measure zero
      \pause\item It is increasing: if $0 \leq x < y \leq 1$, then $f(x) \leq f(y)$
      \pause\item \textbf{It is continuous, but not absolutely continuous}
  \end{itemize}
  
  Mathematically, it can be defined as:
  
  \[ f(x) = \frac{1}{2} \sum_{n=0}^{\infty} \frac{2}{3^n} \chi_{C_n}(x) \]
  
  where $C_n$ is the $n$-th stage in the construction of the Cantor set, 
  and $\chi$ is the characteristic function.  
\end{frame}


\begin{frame}
  \centering
  \includegraphics[scale=0.2]{CantorEscalier-2.png}

  {\footnotesize Source: Wikipedia, the free encyclopedia}
\end{frame}


\begin{frame}{Histogramm or how to understand density}
  
  A \textit{histogram} is a graphical representation of the distribution of a dataset. 

  \begin{itemize}
      \pause\item It is an estimate of the probability distribution of a 
      continuous variable
      \pause\item To construct a histogram, the first step is to "bin" 
      the range of values — that is, divide the entire range of values into 
      a series of intervals
      \pause\item Then count how many values fall into each interval
  \end{itemize}
      
\end{frame}


\begin{frame}
  An example of a histogram:
  
  \begin{center}
  \begin{tikzpicture}
  \begin{axis}[
      ybar,
      enlargelimits=0.15,
      legend style={at={(0.5,-0.15)},
        anchor=north,legend columns=-1},
      ylabel={\# of occurrences},
      symbolic x coords={bin1,bin2,bin3,bin4,bin5},
      xtick=data,
      nodes near coords,
      nodes near coords align={vertical},
      ]
  \addplot coordinates {(bin1,2) (bin2,15)
  (bin3,13) (bin4,7) (bin5,5)};
  \end{axis}
  \end{tikzpicture}
  \end{center}
\end{frame}


\begin{frame}
  \frametitle{Mathematical Expectation}

  We like numbers more than ``strange'' random variables.
  
  \pause
  \begin{definition}
  For a discrete random variable $\xi$ with probability mass function $p(x)$, 
  the mathematical expectation $E\xi$ is defined as:
  \[
  E\xi = \sum_{x} x \cdot p(x)
  \]
  For a continuous random variable $\xi$ with probability density function $p(x)$, 
  the mathematical expectation $E\xi$ is defined as:
  \[
  E\xi = \int_{-\infty}^{\infty} x \cdot p(x) \, dx = 
  \int_{\mathbb{R}} x d P_\xi = \int_{\Omega} \xi d P
  \]
  \end{definition}
  
  \pause
  {\small The mathematical expectation can be thought of 
  as the ``average'' or ``mean'' value of the random variable.}
  
\end{frame}


\begin{frame}{First example}
    $\Omega = \left\{(a_1, \dots, a_n)\mid a_i \in \{0,1\} \right\}, \quad P\left((a_1, \dots, a_n)\right) = \frac{1}{2^n}$
    $$\quad\quad \xi(a_1, \dots, a_n) = a_1 + \dots + a_n$$

  \pause
  \begin{center}
    \begin{tikzpicture}[scale=0.8]
    \begin{axis}[
        ybar,
        enlargelimits=0.15,
        legend style={at={(0.5,-0.15)},
          anchor=north,legend columns=-1},
        ylabel={\# of heads},
        symbolic x coords={$\omega_1$,$\omega_2$,$\dots$,$\omega_{2^n-1}$,$\omega_{2^n}$},
        xtick=data,
        nodes near coords,
        nodes near coords align={vertical},
        ]
    \addplot coordinates {($\omega_1$,2) ($\omega_2$,10)
    ($\dots$,0) ($\omega_{2^n-1}$,7) ($\omega_{2^n}$,5)};
    \end{axis}
    \end{tikzpicture}
    \end{center}
\end{frame}


\begin{frame}{Second example}
  $\Omega$ is unit square, $P(A)$ is area of $A$

  $$\xi(x,y) = |x-y|$$

  \pause $E \xi = ?$
\end{frame}


\begin{frame}{Plot of $\xi(x, y) = |x-y|$}
  \centering
  \begin{tikzpicture}
    \begin{axis}[
        title={},
        xlabel={$x$},
        ylabel={$y$},
        zlabel={$\xi(x, y)$},
        zlabel style={rotate=0},
        view={115}{30},
        grid=major]
    \addplot3 [
        surf,
        shader=interp,
        domain=-2:2,
        domain y=-2:2,
    ] {abs(x-y)};
    \end{axis}
    \end{tikzpicture}
    \pause
    $$E \xi = \int\limits_{\square} |x-y|\,dx\,dy$$
\end{frame}


\begin{frame}{Three Situations for Calculating Expectation}
  $$E\xi = \int\limits_{\mathbb{R}} x d P_\xi$$

  We need to know $P_\xi$
    
  \pause
  If $F_\xi(x)$ is CDF: 
  $$E\xi = \int\limits_{\mathbb{R}} x d F_\xi(x) = 
    \sum\limits_{i=1}^\infty a_ip_i + \int\limits_{\mathbb{R}} x F^\prime_\xi(x) dx,$$
    where $p_i = F_\xi(a_i) - F_\xi(a_i-)$ for the jumps discontinuity $a_i$
  
  \begin{itemize}
    \pause\item If $\xi$ is discrete then $E\xi = \sum\limits_{i=1}^\infty a_ip_i$
    \pause\item If $\xi$ is continuous then $E\xi = \int\limits_{\mathbb{R}} x p(x) dx$
    \pause\item $Ef(\xi) = \sum\limits_{i=1}^\infty f(a_i)p_i + 
    \int\limits_{\mathbb{R}} f(x) F^\prime_\xi(x) dx$
  \end{itemize}

\end{frame}


\begin{frame}
  \frametitle{Expectation Properties}
 
    Let $\xi$ and $\eta$ be random variables, and $a$ and $b$ be constants, then the following properties hold:
    
    \begin{enumerate}
      \item $E[a] = a$, where $a$ is a constant
      \item \textbf{Linearity of Expectation:} $E[a\xi + b\eta] = aE[\xi] + bE[\eta]$
      \pause\item $E[\xi - E\xi] = 0$
      \pause\item If $\xi \geq 0$ a.s. then $E[\xi] \geq 0$
      \item If $\xi \leq \eta$ a.s. then $E[\xi] \leq E[\eta]$
      \item (Jensen's inequality) 
      
      If $\varphi$ is a convex function, then $E[\varphi(\xi)] \geq \varphi(E[\xi])$
      \pause\item (in the future) If $\xi$ and $\eta$ are independent, 
      then $$E[\xi\eta] = E[\xi]E[\eta]$$
    \end{enumerate}
    
\end{frame}


\begin{frame}{Task 1}
  $\Omega = \left\{(a_1, \dots, a_n)\mid a_i \in \{0,1\} \right\}, \quad P\left((a_1, \dots, a_n)\right) = \frac{1}{2^n}$
  $$\quad\quad \xi(a_1, \dots, a_n) = \#(\text{ of occurrences of 01})$$
  $E\xi = ?$

  \pause
  Of course we can calculate 
  $E\xi = \sum\limits_{k=0}^{n-1} k\cdot P(\text{k occurrences})$, but it's hard.

  \pause
  Better to invent new random variables: 
  $$\xi_{ij}(a_1, \dots, a_n) = 
  \begin{cases}
    1,\quad a_ia_j = 01 \\
    0,\quad \text{otherwise}
  \end{cases}$$

  So $E\xi = \pause E \sum\limits_{i} \xi_{ii+1} = \sum\limits_{i} E \xi_{ii+1} = 
  \pause \sum\limits_{i} P(\xi_{ii+1}=1) = \pause\sum\limits_{i} \frac14 = \pause\frac{n-1}{4}$

  % the same scheme works for counting expectation 
  % of the isolated top number for some random graph
\end{frame}


\begin{frame}{Task 2}
  $\Omega = S_n$, choose a random permutation uniformly

  Number is stable, if $k = \sigma(k)$. Random variable $\xi: \Omega \to \mathbb{R}$, find $E\xi$:
  $$\xi(\sigma) = \#(\text{stable numbers}) $$

  \pause
  \begin{block}{Solution}
    Again consider new random variables:
    $$\xi_{k}(\sigma) = 
    \begin{cases}
      1,\quad \text{if } \sigma(k) = k \\
      0,\quad \text{otherwise}
    \end{cases}$$
    
    \pause
    Then $$E\xi = E\xi_1 + \dots + E\xi_n = \sum P(\xi_k = 1) = n\cdot \frac{(n-1)!}{n!} = 1$$
  \end{block}
\end{frame}


\begin{frame}{How far the answers are scattered from expectation?}
  The variance of a random variable $\xi$ is a measure of its dispersion. It is denoted as $Var(\xi)$ or $\sigma^2$ and it is defined as:
  
  \begin{equation*}
  Var(\xi) = E[(\xi - E\xi)^2]
  \end{equation*}
  
  Alternatively, it can be computed using the formula:
  
  \begin{equation*}
  Var(\xi) = E[\xi^2] - (E\xi)^2
  \end{equation*}
  
  The standard deviation of a random variable is the square root of its variance and it is denoted by $\sigma$.

  \pause
  \begin{center}
    \begin{tikzpicture}[scale=1.0]
      % axes
      \draw[->] (-1,0) -- (5,0) node[below right] {$\mathbb{R}$};
      
      % points
      \coordinate (A) at (0,0);
      \coordinate (B) at (2,0);
      \coordinate (C) at (4,0);

      % ticks
      \draw (0, 0.1) -- (0, -0.1);
      \draw (2, 0.1) -- (2, -0.1) node [below] {$E\xi$};
      \draw (4, 0.1) -- (4, -0.1);

      \coordinate [label={$-\sigma(\xi)$}] () at (1, 0.2);
      \coordinate [label={$+\sigma(\xi)$}] () at (3, 0.2);
  
      \draw[dashed] (A) to[out=30,in=150] (B);
      \draw[dashed] (B) to[out=30,in=150] (C);
    \end{tikzpicture}      
  \end{center}

  $Var(a\xi + b) = a^2 Var(\xi)$ -- it can be proven by definition
\end{frame}


\begin{frame}
  \frametitle{Gaussian Distribution}
  
  The Gaussian distribution, also known as the normal distribution, is a continuous probability distribution for a real-valued random variable. Its probability density function is given by:
  
  \begin{equation*}
  f(x) = \frac{1}{\sqrt{2\pi}\sigma}e^{-\frac{(x-\mu)^2}{2\sigma^2}}
  \end{equation*}
  
  where:
  \begin{itemize}
  \pause\item $\mu$ is the mean or expectation of the distribution (and also its median and mode)
  \pause\item $\sigma$ is the standard deviation
  \pause\item $\sigma^2$ is the variance
  \end{itemize}

  \pause
  The standard Gaussian distribution has a mean of 0 and a standard deviation of 1.
  It is denoted as $\xi \sim N(\mu, \sigma^2)$
  
\end{frame}
  

\begin{frame}
  \frametitle{Gaussian Distribution Plot}
  
  \begin{center}
  \begin{tikzpicture}
  \begin{axis}[
      no markers, 
      domain=-3:3, 
      samples=100,
      axis lines*=left, 
      xlabel=$x$,
      ylabel=$f(x)$,
      height=5cm, 
      width=12cm,
      xtick={-3, -2, -1, 0, 1, 2, 3}, 
      ytick=\empty, 
      enlargelimits=false, 
      clip=false, 
      axis on top,
      grid = major,
      axis lines = middle
    ]
    \addplot [fill=cyan!20, draw=none, domain=-3:3] {1/(sqrt(2*pi))*exp(-x^2/2)} \closedcycle;
    \addplot [very thick,cyan!50!black] {1/(sqrt(2*pi))*exp(-x^2/2)};
  \end{axis}
  \end{tikzpicture}
  \end{center}
  
  where $\xi \sim N(\mu=0, \sigma^2=1)$ and:
  \begin{itemize}
  \item $x$ is a normal random variable
  \item $f(x)$ is the probability density function
  \end{itemize}
\end{frame}


\begin{frame}
  \frametitle{What if we have several random variables?}
  $(\Omega, P)$ and $\xi_1, \xi_2, \dots, \xi_k$

  \begin{block}{Question}
    How can we describe all the values?
  \end{block}

  \pause
  \vspace{0.5cm}
  $\vec{\xi}: \Omega \to \mathbb{R}^k, \quad \omega \mapsto (\xi_1(\omega), \xi_2(\omega), \dots, \xi_k(\omega))$

  So we have pair $(\mathbb{R}^k, P_{\vec{\xi}})$ and 
  $P_{\vec{\xi}}(A) = P\left(\left\{\omega \mid \vec{\xi}(\omega) \in A \right\}\right)$ — probability distribution of random vector

  \pause
  \vspace{0.5cm}
  How to define probability $P$ in $\mathbb{R}^k$? Again we need ``good'' sets, 
  and for instance in $\mathbb{R}^2$ it is $\left\{(x,y)\mid x \leq a, y \leq b\right\}$.
  So we need CDF as 
  $$F(x_1, \dots, x_k) = P\left((-\infty, x_1]\times\dots\times(-\infty, x_k]\right)$$

\end{frame}


\begin{frame}
  \frametitle{Task 3}
  Given $F$ in $\mathbb{R}^2$ compute $P\left((a, b]\times(c, d]\right)$.

  \centering
  \begin{tikzpicture}
    \draw[->] (-2,0) -- (1.5,0);
    \draw[->] (0,-2) -- (0,1.5);
      
    \coordinate (ac) at (-1, -1);
    \coordinate (ad) at (-1, 1);
    \coordinate (bc) at (1, -1);
    \coordinate (bd) at (1, 1);
    \filldraw[red] (ac) circle (1pt);
    \filldraw[red] (ad) circle (1pt);
    \filldraw[red] (bc) circle (1pt);
    \filldraw[red] (bd) circle (1pt);
    \coordinate [label={$a$}] () at (-1.17, -0.45);
    \coordinate [label={$b$}] () at (0.83, -0.45);
    \coordinate [label={$c$}] () at (0.15, -1.08);
    \coordinate [label={$d$}] () at (0.15, 0.92);

    \draw (bd) -- (ad) -- (-2,1);
    \draw (bd) -- (bc) -- (1,-2);
    \draw[dashed] (ad) -- (ac) -- (-1,-2);
    \draw[dashed] (bc) -- (ac) -- (-2,-1);
  \end{tikzpicture}        

  \begin{enumerate}
    \pause\item As in 1D we use limit to define the probability of rectangle
    \pause\item There is discrete and continuous random variables
    \pause\item $P_{\vec{\xi}}(A) = \int\limits_{A} p(x_1, \dots, x_n) dx_1\,\dots\,dx_n$
  \end{enumerate}
\end{frame}

\end{document}
