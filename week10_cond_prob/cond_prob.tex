\documentclass[fullscreen=true, bookmarks=true, hyperref={pdfencoding=unicode}]{beamer}

\usepackage[utf8]{inputenc}                                % Кодировка
\usepackage[english,russian]{babel}                        % Переносы
\usepackage{xcolor}                                        % Работа с цветом
\usepackage{amsmath,amssymb,amsfonts}                      % Символы АМО
\usepackage{graphicx}                                      % Графика
\usepackage[labelsep=period]{caption}                      % Разделитель в подписях к рисункам и таблицам
\usepackage{hhline}                                        % Для верстки линий в таблицах
\usepackage[upright]{fourier}
\usepackage{verbatim}
\usepackage{tikz}                                          % Для простых рисунков в документе
\usetikzlibrary{matrix,arrows,decorations.pathmorphing,
shapes.geometric,calc,snakes,backgrounds,arrows.meta,3d}
\usepackage{fancybox}                                      % Пакет для отрисовки рамок
\usepackage{verbatim}                                      % Для вставки кода в презентацию
\usepackage{xmpmulti}                                      % Для вставки gif в презентацию
\usepackage{multirow}

\usepackage{colortbl}
\usepackage{pgfplots}
\pgfplotsset{compat=1.16}
\usepgfplotslibrary{colormaps}

\graphicspath{{../images/}}                                % Путь до рисунков
\setbeamertemplate{caption}[numbered]                      % Включение нумерации рисунков

\definecolor{links}{HTML}{2A1B81}                          % blue for url links
\hypersetup{colorlinks,linkcolor=,urlcolor=links}          % nothing for others

\usetheme{Malmoe}
\usecolortheme{seahorse}

% l' unite
\newcommand{\myunit}{0.5 cm}
\tikzset{
    node style sp/.style={draw,circle,minimum size=\myunit},
    node style ge/.style={circle,minimum size=\myunit},
    arrow style mul/.style={draw,sloped,midway,fill=white},
    arrow style plus/.style={midway,sloped,fill=white},
}
\newcommand{\indep}{\perp \!\!\! \perp}

% \setbeameroption{show notes}
\setbeameroption{hide notes}

\title{Lecture 10. Conditional probability and expectation}
\author{Alex Avdiushenko}
\institute{Neapolis University Paphos}
\date{Bonus-2023}
\titlegraphic{\includegraphics[keepaspectratio,width=0.25\textwidth]{nup_logo.png}}

\setbeamercolor{background canvas}{bg=gray}
\setbeamercolor{normal text}{fg=white}
\setbeamercolor{structure}{fg=cyan}

\begin{document}

\begin{frame}
\transdissolve[duration=0.2]
\titlepage
\end{frame}


\begin{frame}{Conditional Probability Distribution}

  \pause
  \begin{definition}
    Let $\xi$ and $\eta$ be two discrete random variables. Then
    $$ P(\xi = a \mid \eta = b) = 
    \begin{cases}
      \frac{P(\xi = a,\ \eta = b)}{P(\eta = b)} \\
      0, \text{ if } P(\eta = b) = 0
    \end{cases} $$
  \end{definition}
  
  \pause
  Conditional Expectation is
  $$\mathbb{E}(\xi\mid \eta = b) = \sum\limits_i a_i P(\xi = a_i \mid \eta = b) $$

  \pause
  Moreover
  $$\sum\limits_b \mathbb{E}(\xi\mid \eta = b)P(\eta = b) = 
  \sum\limits_{a,b} a P(\xi = a, \eta = b) = \sum\limits_{a} a P(\xi = a) = \mathbb{E} \xi$$

\end{frame}


\begin{frame}{Continuous Distribution}
  \pause
  \begin{definition}
    Let $(\xi, \eta) \sim p_{\xi, \eta}(x, y)$. Then conditional density is
    \[
    p_{\xi \mid \eta = y} (x) = \frac{p_{\xi, \eta}(x, y)}{p_{\eta}(y)}
    \]
  \end{definition}
    
  \pause
  Conditional Expectation is
  $$\mathbb{E}(\xi\mid \eta = y) = \int\limits_{\mathbb{R}} x p_{\xi \mid \eta = y} (x) dx $$

  \pause
  Moreover
  \begin{align*}
    \mathbb{E} \xi = \int\limits_{\mathbb{R^2}} x p_{\xi, \eta}(x, y) dxdy &=
    \int\limits_{\mathbb{R}} dy \left( \int\limits_{\mathbb{R}} x p_{\xi, \eta}(x, y) dx \right) =
    \int\limits_{\mathbb{R}} \mathbb{E}(\xi\mid \eta = y) p_\eta(y) dy = \\
    &= \mathbb{E}_\eta \left(\mathbb{E}(\xi\mid \eta = y)\right)    
  \end{align*}
\end{frame}


\begin{frame}{Example}
  Let $(\Omega, P)$ is probability space and $\xi: \Omega \to \mathbb{R}$, 
  $\chi_A: \Omega \to \mathbb{R}$ — indicator of $A\subseteq \Omega$. 
  Find $\mathbb{E}(\xi\mid \chi_A = y)$.

  \vspace{1cm}
  \pause
  If $X\subseteq \overline{A}$ then
  $$ P(X\mid \chi_A = 0) = \frac{P(X)}{P(\overline{A})} $$

  \pause
  So $$\mathbb{E}(\xi\mid \chi_A = y) = \frac{\mathbb{E}(\xi, \chi_A = y)}{P(\chi_A = y)}$$
\end{frame}


\begin{frame}{Continuous Version of Bayes Formula}
  \begin{definition}
    Let $(\xi, \eta) \sim p_{\xi, \eta}(x, y)$. Then conditional density
    \[
      p_{\xi \mid \eta = y} (x) = p(x\mid y) = \frac{p(y\mid x)p(x)}{p(y)}
    \]
  \end{definition}

\end{frame}


\begin{frame}
  \[
    P(A \cap B \cap C) = P(A \mid B \cap C) P(B \mid C) P(C)
  \]

  \vspace{1cm}
  For $\xi_1, \dots, \xi_n$ we have density $p(x_1, \dots, x_n)$ and
  \[
    p(x_1\ x_2\ \mid\ x_3\ x_4) = \frac{p(x_1\ x_2\ x_3\ x_4)}{p(x_3\ x_4)}
  \]
  and hence
  \[
    p(x_1\ x_2\ x_3) = p(x_1\mid x_2\ x_3) p(x_2\mid x_3) p(x_3)
  \]
\end{frame}


\begin{frame}
  \frametitle{Task 1}
   Suppose we have a dice with $n$ faces, results are uniformly distribute.
   We roll the dice until the sum of all the rolled results is greater than $n$. So
   \begin{align*}
    \Omega &= \left\{ (a_1, \dots, a_k) \mid \sum\limits_{i=1}^k a_i \geq n, \sum\limits_{i=1}^{k-1} a_i < n \right\} \\
    &P(a_1, \dots, a_k) = \frac{1}{n^k} \\
    &\xi(a_1, \dots, a_k) = k \text{ — number of dice rolls},\ \mathbb{E}\xi - ?
   \end{align*}
   
   \pause
   Consider new random variable $\eta_1 = \text{result of first dice roll}$
   \[
    \mathbb{E}\xi = \mathbb{E}_{\eta_1} \left(\mathbb{E}(\xi\mid \eta_1=a)\right) =
    \sum\limits_{k=1}^n \mathbb{E}(\xi\mid \eta_1=k) P(\eta_1 = k)
   \]

   \pause
   Now consider random variables $\xi_m = \text{number of dice rolls to get $m$ in sum}$, so $\xi = \xi_n$.
\end{frame}


\begin{frame}
  \begin{align*}
    &\mathbb{E}(\xi_n\mid \eta_1=k) = \mathbb{E}(1+\xi_{n-k}\mid \eta_1=k) =
    \{\text{as } \xi_{n-k} \indep \eta_1 \} = \mathbb{E}(1 + \xi_{n-k}) \\
    &\Rightarrow \mathbb{E}\xi_m = \frac{\mathbb{E}(1 + \xi_{m-1}) + \mathbb{E}(1 + \xi_{m-2}) + \dots + \mathbb{E}(1 + \xi_{m-n})}{n} =\\
    &= \frac{n + \mathbb{E}\xi_{m-1} \dots + \mathbb{E}\xi_{1}}{n}
  \end{align*}

  \pause
  If $a_k = \mathbb{E}\xi_k,\ a_1 = \mathbb{E}\xi_1 = 1$
  \[
    a_k = 1 + \frac{a_1 + \dots + a_{k-1}}{n} 
    \Rightarrow a_k - a_{k-1} = \frac{a_{k-1}}{n}
    \Rightarrow a_k = \left(1 + \frac{1}{n}\right) a_{k-1} = \left(1 + \frac{1}{n}\right)^{k-1}
  \]
\end{frame}

\begin{frame}
  \frametitle{Task 2: casino}
  \begin{columns}
    \begin{column}{0.5\textwidth}
      \begin{tikzpicture}[scale=1, auto,swap]
        % draw the axes
        \draw[->] (-1,0) -- (4,0) node[right] {$x$};
        \draw[->] (0,-1) -- (0,5) node[above] {$y$};
    
        % draw the line y = N
        \def\N{4} % choose the value for N here
        \draw[dashed, color=blue] (-1,\N) node[above] {$y = N$} -- (4,\N);
    
        % draw the two arrows
        \def\k{2} % choose the value for k here
        \draw[->, color=green] (0,\k) -- (1,\k+1) node[left] {$p$};
        \draw[->, color=red] (0,\k) -- (1,\k-1) node[left] {$1-p$};
        
        % label points
        \node at (0,\k) [circle,fill,inner sep=1pt]{};
        \node[left] at (0,\k) {$(0,k)$};
        
        \node at (1,\k+1) [circle,fill,inner sep=1pt]{};
        \node[above right] at (1,\k+1) {$(1,k+1)$};
        
        \node at (1,\k-1) [circle,fill,inner sep=1pt]{};
        \node[below right] at (1,\k-1) {$(1,k-1)$};
      \end{tikzpicture}        
    \end{column}
    \begin{column}{0.5\textwidth}
      If you are lucky to score $N$, then you win and leave. 
      If you lose all the money to zero, then the game also ends.

      What is $\Omega$ here?
      All the paths!

      \pause
      \begin{block}{Question}
        Find the average length of game $\mathbb{E}\xi_k$.
      \end{block}
    \end{column}
  \end{columns}  
\end{frame}


\begin{frame}
  $\mathbb{E}\xi_0 = \mathbb{E}\xi_N = 0$
  \begin{align*}
    &\mathbb{E}\xi_k = \mathbb{E}(\xi_k \mid \text{win})P(\text{win}) + 
    \mathbb{E}(\xi_k \mid \text{lose})P(\text{lose}) = \\
    &= \mathbb{E}(1 + \xi_{k+1} \mid \text{win})p + 
    \mathbb{E}(1 + \xi_{k-1} \mid \text{lose})q = \\
    &= 1 + p\mathbb{E}(\xi_{k+1}) + q\mathbb{E}(\xi_{k-1})
  \end{align*}

  \pause
  Let column $x = (x_0, \dots, x_N) = (\mathbb{E}\xi_0, \dots, \mathbb{E}\xi_N)$
  \[
    \begin{bmatrix}
    x_0 \\ x_1 \\ \vdots \\ \vdots \\ x_{N-1}\\ x_N
    \end{bmatrix} =
    \begin{bmatrix}
    0 & 0 & \dots & \dots & 0 & 0 \\
    q & 0 & p & 0 & \dots & 0 \\
    0 & q & 0 & p & \dots & 0 \\
    0 & 0 & \ddots & \ddots & \ddots & 0 \\
    0 & \dots & 0 & q & 0 & p \\
    0 & 0 & \dots & \dots & 0 & 0 
  \end{bmatrix} 
  \begin{bmatrix}
    x_0 \\ x_1 \\ \vdots \\ \vdots \\ x_{N-1}\\ x_N    
  \end{bmatrix} +
  \begin{bmatrix}
    0 \\ 1 \\ \vdots \\ \vdots \\ 1\\ 0
  \end{bmatrix} 
  \]

  \pause
  $x = Ax + b \Rightarrow \boxed{x = (I - A)^{-1} b}$
\end{frame}


\begin{frame}{Distribution of sum}
  We can find it only for independent random variables $\xi \indep \eta$, 
  or we must know joint probability distribution.

  \pause
  In discrete case
  $$ P(\xi + \eta = a) = \sum\limits_{u+v = a} P(\xi = u) P(\eta = v) $$
  And for integer values we have convolution
  $$ P(\xi + \eta = k) = \sum\limits_{i \in \mathbb{Z}} p_i q_{k-i}$$
\end{frame}


\begin{frame}
  For continuous case we have $\xi \sim p(x),\ \eta \sim q(y)$
  \begin{align*}
    &F_{\xi+\eta}(t) = P(\xi+\eta \leq t) = \int\limits_{x+y \leq t} p(x)q(y)dxdy = \\
    &= \{x+y = u,\ y = v\} = \int\limits_{u \leq t} p(u-v)q(v)
    \det \begin{bmatrix}
      1 & -1 \\
      0 & 1
    \end{bmatrix} dudv = \\
    &= \int\limits_{-\infty}^t du \int\limits_{\mathbb{R}} p(u-v)q(v) dv 
  \end{align*}

  \begin{align*}
    &F'_{\xi+\eta}(t) = p_{\xi+\eta}(t) = \int\limits_{\mathbb{R}} p(t-v)q(v)dv
    = \int\limits_{\mathbb{R}} p(u)q(t-u)du
  \end{align*}

\end{frame}

\end{document}
