\documentclass{article}
\usepackage[utf8]{inputenc}
\usepackage[T1]{fontenc}
\usepackage{amsmath}
\usepackage{amsfonts}
\usepackage{amssymb}
\usepackage{enumitem}

\usepackage[margin=0.5in]{geometry} % Adjust the margins of document here.

\begin{document}

\section*{Probability and Statistics (12 points max), August 8, 2023}

\begin{enumerate}
  \item (2 points) Explore the system and find the general solution depending on the value of the parameter $\lambda$:
  \begin{align*}
    \begin{cases}
      (1 - \lambda)x_1 + x_2 + x_3 = \lambda \\
      x_1 + (1 - \lambda)x_2 + x_3 = 2\lambda \\
      x_1 + x_2 + (1 - \lambda)x_3 = 3\lambda
    \end{cases}
  \end{align*}

  \textbf{Solution.} Rewrite equations to get 
  $$ x_1 + x_2 + x_3 = (1+x_1)\lambda = (2+x_2)\lambda = (3+x_3)\lambda$$
  If $\lambda = 0$, then $x_1 + x_2 + x_3 = 0$ is the only restriction, else
  $$ x_2 = x_1-1,\ x_3 = x_1-2 $$
  and substituting into the first equation, we get
  $$ x_1 - \lambda x_1 + x_1-1 + x_1-2 = \lambda \Rightarrow x_1 = \frac{3+\lambda}{3-\lambda} $$
  (Notice, that if $\lambda = 3$, than $0 = 6$, what is impossible.)

  \item (2 points) Calculate the characteristic polynomial of the ``N-matrix'':
  \[
    \begin{pmatrix}
      a & 1 & \cdots & 1 & a \\
      a & a & \cdots & 1 & a \\
      \vdots &  & \ddots & \vdots & \vdots \\
      a & 1 & \cdots & a & a \\
      a & 1 & \cdots & 1 & a
      \end{pmatrix} \]

  \textbf{Solution.} We need to calculate determinant of the following matrix.
  Let's substract the first row from the others:
  \[
  \begin{bmatrix}
    a-x & 1 & \cdots & 1 & a \\
    a & a-x & \cdots & 1 & a \\
    \vdots &  & \ddots & \vdots & \vdots \\
    a & 1 & \cdots & a-x & a \\
    a & 1 & \cdots & 1 & a-x
  \end{bmatrix} =
  \begin{bmatrix}
    a-x & 1 & \cdots & 1 & a \\
    x & a-x-1 & \cdots & 0 & 0 \\
    \vdots &  & \ddots & \vdots & \vdots \\
    x & 0 & \cdots & a-x-1 & 0 \\
    x & 0 & \cdots & 0 & -x
  \end{bmatrix}  
  \] 
  and then add $n-2$ intermediate columns, multiplied by $\frac{-x}{a-x-1}$, 
  to the first column:
  \[
    \begin{bmatrix}
      a-x-\frac{x(n-2)}{a-x-1}+a & 1 & \cdots & 1 & a \\
      0 & a-x-1 & \cdots & 0 & 0 \\
      \vdots &  & \ddots & \vdots & \vdots \\
      0 & 0 & \cdots & a-x-1 & 0 \\
      0 & 0 & \cdots & 0 & -x
    \end{bmatrix} = -x(a-x-1)^{n-2}\left(2a - x - \frac{x(n-2)}{a-x-1}\right)
  \]

    
  \item (2 points) Prove, that for any projector matrix (it is also called idempotent) $P$, i.e. $P^2 = P$:
   $$ \operatorname{tr} (P) = \operatorname{rank} (P)$$

   \textbf{Solution.} Observe that an idempotent matrix satisfies the equation 
   $\lambda(1 - \lambda) = 0$. Hence the minimal polynomial is a product of 
   linear factors and the matrix is diagonalizable. 
   Therefore, the rank of the matrix equals the number of non-zero eigenvalues. 
   Since the matrix has eigenvalues 0 or 1, this provides that 
   the trace is equal to the number of unity eigenvalues, 
   or non-zero eigenvalues.

  \item (2 points) Prove, that for odd $n$ there is no $n \times n$ real invertible 
  matrices $A$ and $B$ that satisfy $AB - BA = B^2A$.
  
  \textbf{Solution.} $AB - BA = B^2A \Rightarrow ABA^{-1} = B^2 + B \Rightarrow$ 
  matrices $B$ and $B^2 + B$ are conjugated, in particular, their 
  characteristic polynomials coincide.

  Any polynomial of odd degree has real root, so $B$ has real eigenvalue. 
  Let $\lambda$ be the maximal real eigenvalue of $B$, 
  $\det B \neq 0 \Rightarrow \lambda \neq 0$. 

  Let $v$ be eigenvector: $Bv = \lambda v \Rightarrow (B^2 + B)v = (\lambda^2 + \lambda)v$, i.e. 
  $(\lambda^2 + \lambda)$ is eigenvalue of $B$, but $\lambda^2 + \lambda > \lambda$, 
  contradiction with the maximum $\lambda$.

  \item (2 points) There are a lot of black and white balls and 
  two of them in the bag: one black and one white. 
  We perform the following actions: take out the ball, 
  record its color and put back two balls of the same color. 
  Such actions are performed three times. Considering such 
  an experiment as random, describe the probability space $(\Omega, P)$, 
  that is, the space of elementary outcomes $\Omega$ and 
  the probability measure $P: 2^{\Omega} \rightarrow [0, 1]$ on it. 
  What is the probability of getting a white ball on the third step?

  \textbf{Solution.} It is easy to see that the condition is symmetrical 
  with respect to the permutation of white and black, so the answer is $\frac{1}{2}$.

  \item (2 points) In a square $ABCD$ of area 1, we select a point $M$. 
  Find the expected value of the area of triangle $BCM$.

  \textbf{Solution.} Observe that sum of equally distributed random values $S_{ABM} + S_{BCM} + S_{CDM} + S_{DAM} = 1$. 
  So, due to the linearity of the expectation, we get $\frac{1}{4}$ as the answer.
\end{enumerate}

\newpage
\section*{Probability and Statistics (12 points max), August 8, 2023 — var 2}

\begin{enumerate}
  \item (2 points) Explore the system and find the general solution depending on the value of the parameter $\lambda$:
  \begin{align*}
    \begin{cases}
      (1 - \lambda)x_1 + x_2 + x_3 = 3\lambda \\
      x_1 + (1 - \lambda)x_2 + x_3 = 2\lambda \\
      x_1 + x_2 + (1 - \lambda)x_3 = \lambda
    \end{cases}
  \end{align*}

  \textbf{Solution.} Rewrite equations to get 
  $$ x_1 + x_2 + x_3 = (3+x_1)\lambda = (2+x_2)\lambda = (1+x_3)\lambda$$
  If $\lambda = 0$, then $x_1 + x_2 + x_3 = 0$ is the only restriction, else
  $$ x_2 = x_3-1,\ x_1 = x_3-2 $$
  and substituting into the third equation, we get
  $$ x_3-2 + x_3-1 + x_3 - \lambda x_3 = \lambda \Rightarrow x_3 = \frac{3+\lambda}{3-\lambda} $$
  (Notice, that if $\lambda = 3$, than $0 = 6$, what is impossible.)

  \item (2 points) Calculate the characteristic polynomial of the ``N-matrix'':
  \[
    \begin{pmatrix}
      a & 1 & \cdots & 1 & a \\
      a & a & \cdots & 1 & a \\
      \vdots &  & \ddots & \vdots & \vdots \\
      a & 1 & \cdots & a & a \\
      a & 1 & \cdots & 1 & a
      \end{pmatrix}  \]  

  \item (2 points) Prove, that for any projector matrix (it is also called idempotent) $P$, i.e. $P^2 = P$:
   $$ \operatorname{tr} (P) = \operatorname{rank} (P)$$

  \item (2 points) Prove, that for odd $n$ there is no $n \times n$ real invertible 
  matrices $A$ and $B$ that satisfy $AB - BA = B^2A$.

  \item (2 points) There are a lot of black and white balls and 
  two of them in the bag: one black and one white. 
  We perform the following actions: take out the ball, 
  record its color and put back two balls of the same color. 
  Such actions are performed three times. Considering such 
  an experiment as random, describe the probability space $(\Omega, P)$, 
  that is, the space of elementary outcomes $\Omega$ and 
  the probability measure $P: 2^{\Omega} \rightarrow [0, 1]$ on it. 
  What is the probability of getting a black ball on the third step?

  \item (2 points) In a square $ABCD$ of area 1, we select a point $M$. 
  Find the expected value of the area of triangle $ABM$.

\end{enumerate}

\end{document}
